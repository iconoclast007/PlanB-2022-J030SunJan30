
[3] \footnote{God has done five things for David in the passage, and all five of these things are true in type of the New Testament Christian:
\begin{compactenum}
	\item He has lifted him up (vs. 1).
	\item He has not allowed his foes to rejoice over him. 
	\item He has been healed. 
	\item He has been given physical life as well as life to his soul (vs. 3). 
	\item He has been kept from going to Hell (vs. 3). 
\end{compactenum}
There are “qualifying clauses” on points two and three. Often ``thy foes will rejoice when my sorrow they see, and smile at the tears I have shed.'' The adversaries of the Philippians (Phil. 1:28) were rejoicing over the persecution of the saints, and Paul’s enemies got a ``blessing'' out of him being in jail (Phil. 1:15–-19). You may not have gotten physical healing at salvation, say for an arm, or an eye put out, or a set of teeth gone, or a ruined liver or gall bladder, but you are healed spiritually now; moreover, you will be completely healed physically later (Rom. 8:18–-22) so you will be able to sing Psalm 103:3. \cite{Ruckman1992Psalms} }

[4] \footnote{“Weeping may endure for a night, but....” Again one must be careful. “Joy” has NOT come “in the morning” to many a bed-ridden, disease-plagued saint in a hospital. Weeping often endures a great deal longer than one night. We are on dispensational grounds again, for the context is Paul’s “the sufferings of this present time” (Rom. 8:18) and “our light affliction” (2 Cor. 4:17), etc. Heaven is the “morning,” or the Second Advent is the “morning” (Mal. 4:2). David says he will be satisfied with God’s likeness when he WAKES: that is “the morning.” Jamieson’s use of 2 Samuel 24:15 is a pretty sorry attempt to exegete the text. In other places David speaks of days of mourning and weeping (Ps. 42:3), and don’t you think for a minute that Job got over his “weeping” the day after he buried ten children. He said that tears had been his “meat” day and night; there was no “joy in the morning” for him (Job 3:3, 8). \cite{Ruckman1992Psalms}}

[11] \footnote{\textbf{2 Samuel 6:14} -- And David danced before the LORD with all his might; and David was girded with a linen ephod.} \footnote{\textbf{Isaiah 61:3} -- To appoint unto them that mourn in Zion, to give unto them beauty for ashes, the oil of joy for mourning, the garment of praise for the spirit of heaviness; that they might be called trees of righteousness, the planting of the LORD, that he might be glorified.} \footnote{\textbf{Jeremiah 31:4} -- Again I will build thee, and thou shalt be built, O virgin of Israel: thou shalt again be adorned with thy tabrets, and shalt go forth in the dances of them that make merry.}

[12] \footnote{Exactly as Job 26:1-–4 contains a statement on the Judgment Seat of Christ, so the Psalms often give light on the Pauline epistles. In Psalm 30:11-–12  -— neatly disguised as some historical incident in the life of a dead Old Testament saint --— there lies the clue to WHY GOD SAVED YOU (see Paul in Philippians 3:12-–14). \cite{Ruckman1992Psalms}}





