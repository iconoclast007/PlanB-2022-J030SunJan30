\section{Proverbs 31 Outlines}

\subsection{My Outlines}

\subsubsection{A Different Sort of Woman}
\textbf{Introduction:} Different in Her:
\index[speaker]{Keith Anthony!Proverb 31 (A Different Sort of Woman)}
\index[series]{Proverbs (Keith Anthony)!Pro 31 (A Different Sort of Woman)}
\index[date]{2016/05/31!Proverb 31 (A Different Sort of Woman) (Keith Anthony)}
\begin{compactenum}[I.]
    \item \textbf{Worth} \index[scripture]{Proverbs!Pro 31:10}(Pro 31:10)
    \item \textbf{Work} \index[scripture]{Proverbs!Pro 31:13}(Pro 31:13)
    \item \textbf{Window} \index[scripture]{Proverbs!Pro 31:18}(Pro 31:18) -- she sets a candle, and her place is known as a sanctuary
    \item \textbf{Wardrobe} \index[scripture]{Proverbs!Pro 31:22}(Pro 31:22) -- not dressed for show, but to be about business
    \item \textbf{Words} \index[scripture]{Proverbs!Pro 31:26}(Pro 31:26)
    \item \textbf{Wisdom} \index[scripture]{Proverbs!Pro 31:26}(Pro 31:26)
    \item \textbf{Wholesomeness} \index[scripture]{Proverbs!Pro 31:30}(Pro 31:30)
\end{compactenum}
\textbf{Conclusion:} If you should happen to get one of these, hold on to her!

\subsubsection{Viewing Virtue}
So, here is one attempt at describing this virtuous woman in Proverbs 31. Te Bible has much to say about the opposite, maybe personified as the ``strange woman,'' but for now we this is an opportunity to View Virtue. We see, then:
\index[speaker]{Keith Anthony!Proverb 31 (Viewing Virtue)}
\index[series]{Proverbs (Keith Anthony)!Pro 31 (Viewing Virtue)}
\index[date]{2015/09/12!Proverb 31 (Viewing Virtue) (Keith Anthony)}
\begin{compactenum}
    \item \textbf{Her Purity}\index[scripture]{Proverbs!Pro 31:11} (Pro 31:11) 
    \begin{compactenum}[A.]
        \item Purity in Heart
        \item Purity in Head (her mind and thinking)
        \item Purity in Habit
        \item Purity in her Home (this purity is reflected in what goes on in her home)
    \end{compactenum}
    \item \textbf{Her Prudence} Prudence is the ability or wisdom to see and o what is best in the situation. Godliness adds the requirement for that prudent decision and action to be morally right.
    \begin{compactenum}[A.]
        \item Prudent financially (knows how to be economical - doesn't have a spending problem)
        \item Prudent with time
        \item Prudent in ordering events (what needs to be done first)
    \end{compactenum}
    \item \textbf{Her Perception} \index[scripture]{Proverbs!Pro 31:19} (Pro 31:19) 
    \begin{compactenum}[A.]
        \item She recognizes the needs of others
        \item She sees the need for action
        \item She knows (maybe intuition) the root of the issues - I can think of many areas where I know to trust women's intuition
    \end{compactenum}
    \item \textbf{Her Patience} \index[scripture]{Proverbs!Pro 31:19} (Pro 31:19) 
    \item \textbf{Her Praiseworthiness} \index[scripture]{Proverbs!Pro 31:28} (Pro 31:28) 
    \item \textbf{Her Productivity} \index[scripture]{Proverbs!Pro 31:15, 19} (Pro 31:15, 19) 
    \item \textbf{Her Pricelessness} \index[scripture]{Proverbs!Pro 31:28} (Pro 31:28)  
\end{compactenum}

\subsubsection{Look at this One}
So, here is one attempt at describing this virtuous woman in Proverbs 31. Te Bible has much to say about the opposite, maybe personified as the ``strange woman,'' but for now we this is an opportunity to View Virtue. We see, then:
\index[speaker]{Keith Anthony!Proverb 31 (Look at this One)}
\index[series]{Proverbs (Keith Anthony)!Pro 31 (Look at this One)}
\index[date]{2015/09/12!Proverb 31 (Look at this One) (Keith Anthony)}
 \begin{compactenum}[I.][8]

	\item Her \textbf{Price} \index[scripture]{Proverbs!Pro 31:10} (Pro 31:10)
	\item \textbf{Provides} for Others\index[scripture]{Proverbs!Pro 31:15} (Pro 31:15)
	\item Cares for the \textbf{Poor}\index[scripture]{Proverbs!Pro 31:19} (Pro 31:19)
	\item Has Uncanny \textbf{Perception}\index[scripture]{Proverbs!Pro 31:18} (Pro 31:18)
	\item All about \textbf{Prepartion}\index[scripture]{Proverbs!Pro 31:19} (Pro 31:19)
	\item Cares for the \textbf{Poor}\index[scripture]{Proverbs!Pro 31:19} (Pro 31:19)
	\item Her \textbf{Priority} \index[scripture]{Proverbs!Pro 31:27-28} (Pro 31:27-28)
	\item Worth every  \textbf{Praise} \index[scripture]{Proverbs!Pro 31:27-28} (Pro 31:27-28)
\end{compactenum}


\subsubsection{The Husband of the Virtuous Woman}
In most instances this passage is treated as a list of attributes of the virtuous woman. This is fine, but here, for a change, we'll consider how a husband should respond to a treat the virtuous woman.
\index[speaker]{Keith Anthony!Proverb 31 (The Husband of the Virtuous Woman)}
\index[series]{Proverbs (Keith Anthony)!Pro 31 (The Husband of the Virtuous Woman)}
\index[date]{2015/09/12!Proverb 31 (The Husband of the Virtuous Woman) (Keith Anthony)}
\begin{compactenum}
    \item \textbf{Treasures Her} \index[scripture]{Proverbs!Pro 31:24}(Pro 31:10)
    \item \textbf{Trusts in Her} \index[scripture]{Proverbs!Pro 31:11}(Pro 31:11)
    \item \textbf{Trades by Her} \index[scripture]{Proverbs!Pro 31:14, 24}(Pro 31:14, 24) 
    \item \textbf{Is Taken Care of by Her} \index[scripture]{Proverbs!Pro 31:12,15,21,27}(Pro 31:12, 15, 21, 27) 
    \item \textbf{Triumphs Because of Her} \index[scripture]{Proverbs!Pro 31:23}(Pro 31:23) 
    \item \textbf{Thanks Her} \index[scripture]{Proverbs!Pro 31:28, 30}(Pro 31:28, 30) 
    \item \textbf{He Entrusts Her} The husband of a virtuous woman will entrust her with resources.  She will not be extravagant. She will not be foolish. Instead she will use the resources wisely to be a blessing to her family or others. \index[scripture]{Proverbs!Pro 31:28, 30}(Pro 31:28,30) 
\end{compactenum}

\subsubsection{The Winsome Woman of Wisdom}
Much is said in Proverbs about the strange woman and the foolish woman, Proverbs 31, gives us a picture of the opposite woman, the desirable woman, the one who can be regarded as the epitomy of wisdom.\footnote{13 November 2014, this outline is inspired by a paper written by Tom R. Hawkins in \emph{Bibliotheca Sacra}, January 1996, ``The wife of Noble Character in Proverbs 31:10-31.''}
\index[speaker]{Keith Anthony!Proverb 31 (The Winsome Woman of Wisdom)}
\index[series]{Proverbs (Keith Anthony)!Pro 31 (The Winsome Woman of Wisdom)}
\index[date]{2015/09/12!Proverb 31 (The Winsome Woman of Wisdom) (Keith Anthony)}
\begin{compactenum}
    \item She is like \textbf{Royalty in the Making} - Whether or not the woman spoken of in this chapter was a queen or royalty, she considers herself to be something special and lives above the everyday usual and ordinary.  She is not pretentious or presumptive.  She just lives like everything she has is precious and should be treated as such -- her time, her talent, her tools, her treasure, the things with which she has been entrusted. She dresses, works, lives, like there is meaning in life. \index[scripture]{Proverbs!Pro 31:22, 23} (Pro 31:22, 23) 
    \item She is a \textbf{Rich Mine} \index[scripture]{Proverbs!Pro 31:10} (Pro 31:10) 
    \item She is the \textbf{Role Model} The noble woman presents an image of superior achievement in every area of life. In an era in which industry is discouraged by emphasis on rights over responsibilities, get-rich-quick schemes, and preoccupation with goods acquired on credit, the lyiHa-tw,xe models an industrious and productive lifestyle that contributes to the prosperity of the home and of society at large. In keeping with descriptions throughout the Book of Proverbs, this culminating picture reinforces the thought that anyone whose character, commitment, godliness, and productivity replicate the qualities of this woman has learned to live wisely. \index[scripture]{Proverbs!Pro 31:10-31} (Pro 31:10-31)  
    \item She is a \textbf{Reflection of Maturity} - First, no young bride can possibly fulfill all that is pictured in this poem before developing the maturity that comes only with time. This portrait looks at the finished product, not at a young woman entering marriage. It reflects the cumulative effect of a life lived wisely. At any given point in life a person can only seek to move in the direction this superb and' energetic woman has laid out for all to follow. She serves as d kind of pictorial mentor of the ``ultimate'' wife just as the qualities listed for church leaders in 1 Timothy 3 and Titus 1 describe the direction and focus a man's life should take if he aspires to be a church leader. \index[scripture]{Proverbs!Pro 31:10-31} (Pro 31:10-31) 
    \item Her life should be \textbf{Regarded as a Ministry}  \index[scripture]{Proverbs!Pro 31:10-31} (Pro 31:10-31) 
    \item She is marvelous in the \textbf{Realm of Motherhood}  \index[scripture]{Proverbs!Pro 31:10-31} (Proverbs 31:10-31)  
    \item She is Notable in her \textbf{Rarity as a Mate} - Verse 10 has 13 words, reflecting the rarity of such a woman as this: (1) rebellion of females in actually wanting to be or being such a woman, (2) rebellion of males in wanting and finding such a woman (or waiting for and recognizing such), and (3) since there is a role in the husband in developing such a woman over time, the rebellion of husbands against doing so. \index[scripture]{Proverbs!Pro 31:10-31} (Pro 31:10-31) 
\end{compactenum}

